%clase de texto matemático
% Clase del documento
\documentclass[a5paper, 12pt]{article}

% Paquetes
\usepackage[utf8]{inputenc}
\usepackage[spanish]{babel}
\usepackage[top=2cm, left=2cm]{geometry}
\usepackage{amsmath, amssymb, amsfonts, latexsym}
\usepackage{graphicx}
\usepackage{color}
\usepackage{multicol}
\usepackage{nicefrac}
\usepackage{ marvosym }
% Comandos
\parindent = 0mm

\author{Kevin}

\title{Clase De Texto Matemático}

\date{\today}

% Contenido

\begin{document}
\maketitle
 La optimizacion de funciones no es un tema analizado únicamente con herramientas del calculo en una variable y de la programación lineal. esta se puede generealizar a espacios mas generales como los espacion de Banach. A continuacion se presenta el siguiente problema de optimizacion:
 
 \[
 minJ(u,y,a)=\int_{0}^{a} (u^,(x))^2 dx +  \int_{0}^{a} y(x)^2dx + \dfrac{a^2}{med(0,a,a,a^2)^,}
\]

sujeta a
\[-u^,,(x) + \alpha(x)u(x)=y(x) en (0,a),\]
\[u=0 en{0,a},\]
\[\displaystyle\lim_{x\to a} f(a)\]
\[a\geq4\]




\end{document}