%clase de documento
\ documentclass [a5paper, doc, 12pt, apacite] {} apa6

 %paquetes
\ usepackage [UTF-8] {} inputenc
\ usepackage [español, es-nolayout, es-nodecimaldot, es-tabla] {babel}
\ usepackage [T1] {} fontenc
\ usepackage [T1] {} fontenc


%comandos
\ author {Elisa Kiseljak}
\ title {} epílogo
\ shorttitle {}
\ date { \ today }
\ geometría {top = 2,5 cm, inferior = 2cm, izquierda = 2cm, derecha = 2cm}
\ pagenumbering {} romana


%contenido
\ begin {document}
\ maketitle
% Europeas Historias Tres
\ afiliación {} Tres Historias Europeas
\ input {text}
La vida ESTA. Absurda Recorre el pecado ESPEJOS el trazado de su laberinto propio. Y avanza, temerosa del cielo y de la tierra, de su propio reflejo Sobre el velo, DEL AGUA QUE ESCONDE los recuerdos.

La vida, ESTA. Trata de reconocerse en los Fragmentos desguazados de la luna, Atrapados con cazamariposas Gigantes Que nadie es capaz de sostener en vuelo sin cubrirse los ojos de la arena.

La vida, ESTA. medios Atrofiada Ya en EL REPARTO de patines, de alas, de olfatos, de espadas, madrigueras. Asustada camina caminos en silencio. Olvidando las estatuas de sal en las esquinas, las Mujeres Muertas de los Arboles, Los Pozos secos de su voz aguda. tiembla Y.

La vida, ESTA. Se Detiene una descansar junto a Las Puertas Y olvida innecesarias Contraseñas, imprescindibles para Hacerse Pequeña y Atravesar paredes desconchadas de goteras de alientos abatidos.

La vida, ESTA. Sonora y agotada se derrumba en fosas de ruinas dibujadas con paletas de colores de madera. Y busca Una Estrella Muerta En La Que El Corazón esconder los antes de irse. Suspicaz ante el atraso y la Memoria. Gelida Lamentos ante seductores. Chiquita ante pieles expuestas en los rincones de las Orillas de los Mares Recuerda que. NOSTALGICA sucumbe ante su propio miedo. La vida, ESTA.

Sentada, reposa y espera, con Sus Ojos de Puñal clavados en la arena, ansiosa, atenta. Buscando any nube Que la quiera, Que la esconda, la proteja. Y Suspira, cansada y ya dormida, frente un Imágenes Que empieza a Hacer de piedra. Soñando Un beso, un beso en solitario, Una gruta ENTERA de belleza.


%bibliografia
% Codificación: UTF-8
\ maketitle
\ bibliographystyle {} apacite
\ bibliografía {} Libro
\ nocite {} Bosch
@book {Bosch,
  title = {historias Europeas Tres},
  publisher = {} Caballo de Troya,
  año = {2005},
  autor = {} Lolita Bosch,
  Dirección = {} Madrid,
  ISBN 84-934195-4-0 = {},
}

@Comment {JabRef-meta: databaseType: bibtex;}
\ maketitle
\ end {document}